\documentclass[10pt,a4paper]{report}
\usepackage[latin1]{inputenc}
\usepackage{amsmath}
\usepackage{amsfonts}
\usepackage{amssymb}
\usepackage{graphicx}
\author{AKOLL CARLTON NOEL}
\title{MAKERERE UNIVERESITY STUDENTS? RESULTS TRACKING SYSTEM RESEARCH PROJECT.}
\begin{document}
	\begin{flushleft}
		\subsection*{INTRODUCTION}This is a predominately applied research to help tackle the issue of results complaints for the students of Makerere University.\\
	
\subsection*{PROBLEM STATEMENT}This system entails all issues of results complaints like missing exams marks, missing coursework marks and test marks and even wrong marks.  Makerere University's current students' result complaint system is quite underwhelming and very hectic to say the least. The sources of this research come from personal experience and also from peers studying in Makerere University. This concept research project is to help simplify, build and improve the current students' results tracking system. \\

\subsection*{RESEARCH BACKGROUND}The student population of Makerere University is very large and inevitably students' results get either misplaced or are simply marked incorrectly due to lecture fatigue from marking many exam scripts. Human error also by the staff who enter the students' results in the current system because of confusion from entering too many results or just simple mistakes in typing in the wrong marks for a particular student. The students' complaint systems involves a manual process where the student with a complaint fills a forms, meets up with lecturers, registrar and constantly keeps up with involved actors in order to get the results rectified which is by any means tremendously tedious.
Inquiries cannot be readily answered since the current system can't keep track of every results complaint from each student and updating them individually of the progress. With the current process involved and the mounting frustrations and discomfort from students, there is an urgent need to develop an online University missing marks complaint system.\\

\subsection*{AIMS AND OBJECTIVES}From the explanations above to simply state the aims and objectives of this research project are to develop a system that provides an interface where students of the University of Makerere can be able to file a complaint about their problem concerning their results by filling forms. These forms are then linked to a back-end database where each student is uniquely identified by their student number and registration number. This complaint can then be followed up promptly by a results' complaint attendant remotely over the same interface online thereby simplifying and easing the work of both the student and the staff.\\

\subsection*{SCOPE}The system will have the ability to notify students of the progress of their complaint, how far it has reached and to which office the complaint is in at that moment when the student logs into the system. This system will also be able to store records of the student's complaints for future reference. The system will have the capability of capturing there student complaints online without of benching at different offices so that any student with a complaint will be able to submit his or her complaint at the convenience of their laptop or smart phone. The system is going to have the ability to provide accurate information to a student about the progress of their complaint, unlike the current system where by a student can submit in their complaint and lose track of what really is going on because of lack of feedback.  


\end{flushleft}

\end{document}